\documentclass{beamer}

\usetheme{beamer_tu}
\usepackage{lmodern}
\usepackage{fontspec}
\usepackage[spanish]{babel}
\usepackage[scale=2]{ccicons}
\usepackage{graphicx}
\usepackage{tikz}


\title{\textcolor{normal}{Beamer Transforma Uruguay}}
\subtitle{}
\date{\today}
\author{Paula Pereda}
\institute{\url{http://github.com/transformauy}}
\begin{document}

{
  \usebackgroundtemplate{\includegraphics[width=\paperwidth]{img/caratula.jpg}}
  \begin{frame}
   \vspace*{80pt}
   \hspace*{1em}
\titlepage
  \end{frame}
}

\setbeamertemplate{background canvas}{\includegraphics[width=\paperwidth]{img/fondo.jpg}}

\begin{frame}{Ejemplo}
  \framesubtitle{Un template de beamer de Transforma Uruguay}

  \texttt{beamer\_tu} incluye: 

  \begin{columns}
    \column{.5\textwidth}
      \begin{itemize}
        \item la \alert{paleta de colores} de la marca
        \item imagen de carátula
        \item imagen de fondo
      \end{itemize}

    \column{.5\textwidth}
      \begin{block}{Sobre la paleta...}
         los colores son: verde, naranja, gris y azul
      \end{block}
  \end{columns}
  
\end{frame}

\begin{frame}{Licencia}

  \begin{block}{Se puede obtener la fuente del template y una presentación demo aquí:}

  \begin{center}\url{https://github.com/transformauy/simple_beamer}\end{center}

  \end{block}
  
  \begin{block}{El tema está basado en el tema 'simple' de Facundo Muñoz}
  \end{block}
\end{frame}

\end{document}